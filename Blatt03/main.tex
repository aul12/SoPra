\documentclass[DIN, pagenumber=false, fontsize=11pt, parskip=half]{scrartcl}

\usepackage{amsmath}
\usepackage{amsfonts}
\usepackage{amssymb}
\usepackage{enumitem}
\usepackage[utf8]{inputenc} % this is needed for umlauts
\usepackage[ngerman]{babel} % this is needed for umlauts
\usepackage[T1]{fontenc} 
\usepackage{commath}
\usepackage{xcolor}
\usepackage{booktabs}
\usepackage{float}
\usepackage{tikz-timing}
\usepackage{tikz}
\usepackage{pgf-umlsd}
\usepgflibrary{arrows} % for pgf-umlsd
%\usepackage{verbatim}
\usepackage{multirow}
\usepackage[final]{pdfpages}

\usetikzlibrary{calc,shapes.multipart,chains,arrows}

\title{Softwaregrundprojekt}
\author{Paul Nykiel}

\begin{document}
    \maketitle
    \section{Anwendungsfälle}
    \subsection{Anwendungsfälle}
    Die Anforderungen sind der Musterlösung entnommen.
    \subsubsection{Hauptmenü}
    Beeinhaltet:
    \begin{itemize}
            \item FA1: Hauptmenü
    \end{itemize}
    \subsubsection{Hilfe}
    Beeinhaltet:
    \begin{itemize}
            \item FA2: Hilfe
    \end{itemize}
    \subsubsection{GameOver}
    Beeinhaltet:
    \begin{itemize}
            \item FA6: GameOver 
    \end{itemize}
    \subsubsection{Highscore}
    \begin{itemize}
        \item FA3: Highscore (Ansicht)
        \item FA4: Highscore (Dateiformat)
        \item FA5: Highscore (lesen \& schreiben)
    \end{itemize}
    \subsubsection{Spiel}
    Beeinhaltet:
    \begin{itemize}
        \item FA7: Spielfeld (Canvas)
        \item FA8: Spielfeld (Metrik)
        \item FA9: Spielfigur
        \item FA10: Hindernis (Turm)
        \item FA11: Hindernis (Dementoren)
        \item FA12: Beschleunigung der Spielfigure nach oben
        \item FA13: Laden von Grafiken
        \item FA14: Hitboxen
        \item FA15: Generierung von Hindernissen
        \item FA16: Items (Allgemein)
        \item FA17: Item (Turbo Mode)
        \item FA18: Item (Unverwundbarkeit)
        \item FA19: Spielfigur (Fallbeschleunigung)
    \end{itemize}
    \includepdf[pages=-,scale=1,pagecommand={\subsection{Anwendungsfalldiagramm}}]{anwendungsfall.pdf}
    \includepdf[pages=-,scale=1,pagecommand={\section{Zustandsdiagramm}}]{zustandsdiagramm.pdf}
    \section{Sequenzdiagramm}
    \begin{figure}[H]
        \centering
        \begin{sequencediagram}
            \newinst{c1}{Spieler 1}
            \newinst{c2}{Spieler 2}
            \newinst[2]{s}{Server}

            \begin{call}
                {c1}{submitPoints(points, seed, path)}
                {s}{isNewHighscore}
            \end{call}
            \begin{call}
                {c2}{createGhostGame()}
                {s}{gameInstance}
            \end{call}
            \mess{c2}{startGame}{s}
            \mess[2]{s}{dataStream}{c2}
            \begin{messcall}
                {c2}{gameOver(points)}{s}
            \end{messcall}
        \end{sequencediagram}
    \end{figure}
\end{document}
