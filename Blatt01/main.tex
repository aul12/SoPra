\documentclass[DIN, pagenumber=false, fontsize=11pt, parskip=half]{scrartcl}

\usepackage{amsmath}
\usepackage{amsfonts}
\usepackage{amssymb}
\usepackage{enumitem}
\usepackage[utf8]{inputenc} % this is needed for umlauts
\usepackage[ngerman]{babel} % this is needed for umlauts
\usepackage[T1]{fontenc} 
\usepackage{commath}
\usepackage{xcolor}
\usepackage{booktabs}
\usepackage{float}
\usepackage{tikz-timing}
\usepackage{tikz}
\usepackage{multirow}

\usetikzlibrary{calc,shapes.multipart,chains,arrows}

\title{Softwaregrundprojekt}
\author{Paul Nykiel}

\newcommand{\anforderung}[5] {
    \begin{table}[H]
        \centering
        \begin{tabular}{p{4cm}|p{10cm}}
            \toprule \\
            ID & #1 \\
            \midrule \\
            TITEL & #2 \\
            BESCHREIBUNG & #3 \\
            BEGRÜNDUNG & #4 \\
            ABHÄNGIGKEITEN & #5 \\
            \bottomrule
        \end{tabular}
    \end{table}
}

\newcounter{fanforderungCount}
\newcommand{\fanforderung}[4] {
    \stepcounter{fanforderungCount}
    \anforderung{FA\thefanforderungCount}{#1}{#2}{#3}{#4}
}
\newcounter{nfanforderungCount}
\newcommand{\nfanforderung}[4] {
    \stepcounter{nfanforderungCount}
    \anforderung{QA\thenfanforderungCount}{#1}{#2}{#3}{#4}
}


\begin{document}
    \maketitle
    \section{Funktionale Anforderungen}
    \fanforderung{Hauptmenü}
               {Nach dem Anwendungsstart wird dem Benutzer das Hauptmenu angezeigt. Der Benutzer kann folgende Aktionen im Hauptmenü ausführen:
               \begin{enumerate} 
                   \item Spiel starten 
                   \item Hilfe anzeigen 
                   \item Highscore anzeigen 
                   \item Spiel beenden 
               \end{enumerate}
               }
               {Damit der Benutzer alle Bestandteile der Anwendung über einen Dialog erreichen kann}
               {}
    \fanforderung{Beschleunigung der Spielfigur nach oben}{Der Benutzer kann durch (mehrfaches) Drücken der Leertaste die Spielfigur nach oben (y-Achse) beschleunigen. Je ein Tastendruck gibt einen beschleunigenden Schub nach oben, durch halten der Leertaste wird maximal ein Schub ausgelöst.}{}{}
    \fanforderung{Hilfemenü}{Beschr}{Begr}{Abh}
    \fanforderung{Highscore anzeigen}{Beschr}{Begr}{Abh}
    \fanforderung{2D}{Beschr}{Begr}{Abh}
    \fanforderung{Side-Scroller}{}{}{}
    \fanforderung{Hindernisse }{}{}{}
    \fanforderung{Auf die Spielfigur wirkt dauerhaft eine nach unten beschleunigende Kraft.}{}{}{}
    \fanforderung{Falls ein Hindernis berührt wird führt dies in der Regel zum Scheitern.}{}{}{}
    \fanforderung{Die Geschwindigkeit, mit welcher die Hindernisse sich auf die Spielfigur zu bewegen, wird kontinuierlich gesteigert}{}{}{}
    \fanforderung{Zufällige Items}{}{}{}
    \fanforderung{Item einsammeln}{}{}{}
    \fanforderung{Items deaktivieren}{}{}{}
    \fanforderung{Item: Unverwundbarkeit}{}{}{}
    \fanforderung{Item: Turbo Mode}{}{}{}
    \fanforderung{Item: Doppelte Punkte}{}{}{}
    \fanforderung{Item: Troll}{}{}{}

    \section{Nichtfunktionale Anforderungen}
    \nfanforderung{Robustheit}{Die Anwendung darf nicht abstürzen. Bei 100 Spielen darf maximal 1 Spiel aufgrund eines Fehlers abgebrochen werden müssen.}{}{}
    \nfanforderung{Java}{Beschr}{Begr}{Abh}
    \nfanforderung{Highscore speichern}{}{}{}
    \nfanforderung{30 Frames pro Sekunde}{}{}{}
\end{document}
